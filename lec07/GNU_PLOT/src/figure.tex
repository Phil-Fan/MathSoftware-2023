\documentclass[12pt, a4paper, oneside]{ctexart}
\usepackage{amsmath,caption, amsthm, amssymb, bm, graphicx, hyperref, mathrsfs,subfig,cite}

\title{Assignment7 Generate Pictures}
\author{PhilFan 樊铧纬}
\date{\today}
\linespread{1.5}
\newcounter{problemname}
\newenvironment{problem}{\stepcounter{problemname}\par\noindent\textbf{题目\arabic{problemname}. }}{\\\par}
\newenvironment{solution}{\par\noindent\textbf{解答. }}{\\\par}
\newenvironment{note}{\par\noindent\textbf{题目\arabic{problemname}的注记. }}{\\\par}


\begin{document}
\maketitle
在CC98开了一个记录的楼,记录学习中遇到的问题和学习过程;~\cite{ref1}


\begin{problem}
	绘制三维椭圆
\end{problem}

\begin{solution}
	\setcounter{section}{1}
	\paragraph{生成过程}


\begin{figure}[htbp]
	\centering
	\includegraphics[width=.8\textwidth]{figure.png} %1.png是图片文件的相对路径
	\caption{生成的图片} %caption是图片的标题
	\label{img} %此处的label相当于一个图片的专属标志,目的是方便上下文的引用
\end{figure}
由用户指定了椭圆的长轴和短轴,再由gnuplot生成图片后,输出到当前目录下的figure.png, 再引用到当前tex文件中生成最终的pdf
	\ref{img}
\end{solution}

% 使用 BibTeX 管理文献引用
\bibliographystyle{plain}
\bibliography{references}

\end{document}


